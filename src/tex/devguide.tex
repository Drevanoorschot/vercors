\documentclass{report}

\usepackage{todonotes}
\usepackage{enumitem}
\usepackage{fullpage}
\usepackage{amssymb}
\usepackage{hyperref}
\usepackage{jmllistings}
\lstset{language=[jml]java,numbers=left,escapechar=\#}

\usepackage{tabularx}

\usepackage{vercors-macros}

\usepackage{float}
\usepackage{enumitem}
\floatstyle{plain}
\newfloat{listing}{tp}{lst}%%[section]
\floatname{listing}{\bf Listing}
%\newenvironment{listing}{\begin{figure}}{\end{figure}}
%\newenvironment{listing*}{\begin{figure*}}{\end{figure*}}
\newcommand{\listref}[1]{Lst.~\ref{#1}}
\newcommand{\Listref}[1]{Listing~\ref{#1}}

\author{Stefan Blom}
\title{Developers Guide to the VerCors Toolset}

\begin{document}

\maketitle

\tableofcontents

\chapter{Overview}

The VerCors tool is a compiler for specified programs. It works by 
parsing the inputs, applying many small program transformations,
then it can output the result and/or invoke an external verifier on them.
If a verifier is invoked then the error messages are translated back
and reported as the corresponding cause in the original program.

\chapter{Structure}

The VerCors sources have been divided into several modules.
These modules are separate because they track external
projects and or may be reusable in different contexts.

First, we will give an overview of the modules.
Second, we will describe how to configure Eclipse to develop
(parts of) the VerCors tool set.

\section{Modules that are part of the VerCors Tool}

\begin{description}
\item[HRE] The Hybrid Runtime Environment contains the classes that can potentially be
reused by projects that have nothing to do with program verification. For example,
the classes that are used for the following tasks are located in HRE:
\begin{itemize}
\item Running command line tools.
\item Text output (logging) management.
\item Generic debug utilities.
\item Command line parsing and configurations.
\item Plugin infra-structure based on class loaders.
\end{itemize}
\item[Viper API]
This project contains the Java API that is used to access the functionality of the Viper project.
\item[core] This is where most of the implementation of VerCors lives:
\begin{itemize}
\item parsers/pretty printers
\item program transformers
\item backends
\end{itemize}
\item[main] For historic reasons the main program had to be in a separate project.
Thus in this prohect, one may find:
\begin{itemize}
\item The VerCors main program.
\item The command line test infrastructure.
\end{itemize}
The first could be moved into the core project. The second could be refactored into
parts that belong in the HRE projects and parts that belong in core.
\end{description}

\section{Modules that are part of the VerCors Tool Viper plugins.}

\begin{description}
\item[Viper API]
This project contains the Java API that is used to access the functionality of the Viper project.
Note that an instance of a plugin will only work for an instance of a tool if both have been
linked using the same version of the API.
\item[Silver]
This project tracks the Viper silver module, which contains the Sivler parser and AST.
It has been extended with implementations of the Viper API to allow interfacing with
the VerCors Tool.
\item[Carbon]
This project tracks the Viper Carbon module, which implements the Carbon verifier.
It has been extended with a basic implementation of the Viper API.
\item[Silicon]
This project tracks the Viper Silicon module, which implements the Silicon verifier.
It has been extended with an advanced implementation of the Viper API.
\end{description}

\section{Setting up for developping with Eclipse.}

In order to avoid having to specify the entire list of dependencies of the Viper project,
we simply use the jars assembled for Carbon and Silicon to provide the dependencies.
Thus, the first step of setting up is to perform a command line build.
Also, if you are going to develop any of the Viper modules, you will
need to install the Scala IDE plugin for Eclispe.
Afterwards, several sub-directories are used as the (non-standard) locations of eclipse projects.


\begin{enumerate}
\item Install the Scala IDE plugin for Eclipse.
\item Perform a command line build:
\begin{verbatim}
> ant
\end{verbatim}
\item Create a Java project HRE with the sub-directory hre as its root.
\item Create a Java project Viper API with the sub-directory viper/viper-api as its root.
\item Create a Java project VCT core with the sub-directory core as its root.
\\
Add the following dependencies:
\begin{itemize}
\item project HRE
\item project Viper API
\end{itemize}
\item Create a Java project VCT main with the sub-directory main as its root.
\\
Add the following dependencies:
\begin{itemize}
\item project HRE
\item project VCT core
\end{itemize}
\item Create a Scala project Silver with the sub-directory viper/silver as its root.
\\
Further configuration is needed, but will be performed in a later step.
\item Create a Scala project Silicon with the sub-directory viper/silicon as its root.
\\
Further configuration is needed, but will be performed in a later step.
\item Create a Scala project Carbon with the sub-directory viper/carbon as its root.
\\
Further configuration is needed, but will be performed in a later step.
\item Fix the setup for Silver:
\begin{verbatim}
sources:
  silver/src/main/scala

output:
  Silver/target/scala-2.11/classes

dependencies:
  project Viper API
  Silicon/target/scala-2.11/silicon.jar
  Java
\end{verbatim}
\item Fix setup for Silicon:
\begin{verbatim}
sources:
  silicon/src/main/scala

output:
  Silicon/target/scala-2.11/classes

dependencies:
  project Viper API
  Silicon/target/scala-2.11/silicon.jar
  Java
\end{verbatim}
\item Fix setup for Carbon:
\begin{verbatim}
sources:
  carbon/src/main/scala

output:
  Carbon/target/scala-2.11/classes

dependencies:
  project Viper API
  Carbon/target/scala-2.11/carbon.jar
  Java
\end{verbatim}
\end{enumerate}





\chapter{Parsers}

The VerCors tool itself provides parsers for three languages:
PVL\footnote{Program Verification Language / Prototype Verification Language}, Java and C.
It uses the parser in the Viper library to be able to parse Silver.

The parsers for C and Java work in two passes. The first pass parses the executable code
and turn it into an AST with everything else stored as comments. The second pass
parses the comments and replaces them with specifications and ghost code ASTs.

The built-in parsers use ANTLR-v4 grammars. These grammars are designed
to keep the amount of duplication as small as possible. That is, we use
the import mechanism of ANTLR-v4 to be able to re-use
existing ANTLR-v4 grammars for Java and C. We also use a common specification
language grammar for all languages.

\par\noindent\begin{tabularx}{\textwidth}{lX}
grammar & description
\\
C.g4 & Imported grammar for C.
\\
Java.g4 & Imported grammar for Java.
\\
val.g4 & VerCors Annotation Language grammar.
\\
CML.g4 & Combine C and val to get specified C.
\\
JavaJML.g4 & Combine Java and val to get specified Java.
\\
PVFull.g4 & Define the specified language PVL around val.
\end{tabularx}

The existing grammars for Java and C are modified in the following ways:
\begin{itemize}
\item Comments are assigned to a channel in order to preserve them.
\item Line direction annotations, as generated by the C pre-processor are
assigned to a channel in order to compute the correct file positions of specifications.
\item The notion of identifier is modified. Both to allow val keywords to be used as identifiers
and to allow the use of \verb+\result+, etc.
\item The operators \verb+**+, \verb+==>+, and \verb+-*+ are inserted into the expression
syntax.
\item Several extra primary expression are added to the language.
\item Several extra primary types are added.
\item Additional statements are added.
\item Additional declarations are added.
\end{itemize}
The resulting grammars are used for both parsing the main file and for parsing
the comments containing the specifications.

The VerCors library uses its own intermediate language: Common Object Language (COL).
This is not a well-designed language, but rather a collection of AST nodes
that allows elegant representations of the various languages supported and
used by the tool: C, Java, PVL, JML, Silver, Chalice, Boogie, Dafny, etc.

The parse trees produced by the ANTLT-v4 parser are converted into COL
using visitors. Rather than handwriting all cases for every visitor
for every language variation, we use a layered class hierarchy.
The root is the \lstinline+ANTLRtoCOL+ class. This class takes care
of the elements of the common part of the specification language
and can be parameterized to take care of many expressions and
statements that are common in languages. It can do so because 
it is parameterized with a \lstinline+Syntax+ object, in which the 
syntax of the operators and built-in functions of COL can be found.
An optional middle layer consists of \lstinline+AbstractLanguageToCOL+ classes,
which implement the conversion of common parts of langauges.
The leafs contain the clauses for the actual language variant.
(Java, specified Java, etc.) To keep the amount of work needed to implement
these leaf classes minimal, the visitors are invoked using utility function that
will first call the visitor to allow the visitor to return the correct conversion,
but when the visitor returns null (which is the default implementation Eclipse generates
with one mouse click) a default conversion will be attempted.

\section{Specification Language}

The specification language contains block constructs, such as atomic blocks and action blocks.
Rather than allowing these blocks to begin and end in arbitrary locations, we haven chosen to
build upon the native block structure of the supported programming languages. Thus, the atomic
block
\begin{lstlisting}
action swap {
  int tmp = x;
  x = y;
  y = tmp; 
}
\end{lstlisting}
can be written in Java and C as
\begin{lstlisting}
{ //@ action swap ;
  int tmp = x;
  x = y;
  y = tmp; 
}
\end{lstlisting}

\chapter{Error Reporting}

The error reporting mechanism is built around the notion of an \emph{origin}.
Every transformation maintains the origin of every AST node that it produces.
This origin contains enough information to both trace back every AST node
to the source code from which it was generated and to convert error
messages with respect to the output to the correct message for the input.
E.g. an assertion failed in the output, might be reported as a
post-condition failure at a return statement.


\section{Implementation}

The class \lstinline+vct.logging.VerCorsError+ captures the essence of the
VerCors error model. The ErrorCode, which tells what was wrong and the
SubCode saying why. Many of the encodings of the VerCors toolset
encode constructs and/or proof obligations in various ways.

This means that the list of errors with respect to the output of every pass
has to be translated to be a list of error with respect to the input of the
pass. The simplest ways of implementing this, is to call the
method \lstinline+set_branch+ on generated code for a certain proof obligation
and, whose origin has to be the location where the error should be reported.
In addition an \lstinline+vct.logging.ErrorMapping+ has to be maintained
which performs a simple lookup per branch of errors.

For example, when a return statement is converted to a sequence
of an assertion of the post-condition and then an assume false,
the assertion failed error is translated to a postcondition failed.




\chapter{The Compiler Passes}

\section{Magic Wand Encoding}

The magic wand encoding pass:
\begin{itemize}
\item Replaces magic wand formulas with abstract predicates.
\item Replaces magic wand creation blocks with a block that exhales the 
used permissions and formulas and inhales the magic wand,
while at the same time generating a method that check the proof
of the magic wand provided in the create block.
\item Replaces applications of magic wands with the exhales of the 
magic wand and the left-hand side, followed by the inhale of the right-hand side.
\end{itemize}

\subsection{ordering}

The magic wand pass create predicates that have multiple arguments.
Thus, this pass has to be applied before applying the witness encoding.
To enable the witness encodign afterwards, any witness label given to a
magic wand is copied to the generated magic wand predicate.


\section{Witness Encoding}

The witness encoding pass allows backends that do not support arbitrary
predicate parameters to reason about specification with such predicates,
by encodign these predicates as witness objects that are explicitly
manipulated.

\subsection{ordering}

This pass has to be applied after any pass that can create predicates
with arbitrary parameters.


\subsection{effect on other passes}

Every pass that is applied before this pass has to preserve the
witness handling instructions and generate witness handling instructions
for any new code it inserts.


\subsection{limitations}

The current implementation requires witnesses to be used for all predicates that
are not inlined. By tagging predicates that neither have parameters nor call
any predicates with parameters, the tagged predicates could be excluded
from the witness bookkeeping.



\chapter{Program Conversion by Rewriting of Expressions}

Simplifying logical formulas can transform them from a large subset to
a smaller subset that is compatible with the limitations of a given back-end
and/or improve the performance of a back-end by standardizing formulas.
To this end, a rewriter is under development that can read a rewrite system
from a file and apply this rewrite system to all (specification) expressions
in a program.

Roughly, the idea is that a rewrite system is represented as a ghost class,
in which:
\begin{itemize}
\item Fields in the class can be used as variables in the rules.
\item Binding constructs, such as the \lstinline+\forall+ are allowed.
\item The non-occurrence of a bound variable on the left-hand side of a rule
is written \lstinline+(e!i)+, where \lstinline+e+ has to be a field and
\lstinline+i+ has to be a bound variable.
\item Substitution on the right-hand side is written with \lstinline+\let+.
\end{itemize}

Currently, there are several bugs and limitations in the rewriter:
\begin{enumerate}
\item Types are not matched, so a rewrite rule meant for arrays get applied to sequences too,
which sometimes works, but often leads to errors.
\item The origins of the rewritten nodes are not set properly, so error messages are
reported at completely bogus locations.
\end{enumerate}

%\cleardoublepage
%\bibliographystyle{abbrv}
%\bibliography{extra}

\appendix

\chapter{Keywords}

\section*{Special types}
\begin{description}
\item[frac] non-zero fractions
\item[zfrac] zero-able fractions
\item[resource] type of resource expressions
\item[classtype] type of type expressions
\item[seq$<$E$>$] Seqeunce or list of E
\item[set$<$E$>$] Set of E
\item[bag$<$E$>$] Bag of E
\item[cell$<$E$>$] Object that contains a single E with set and get (internal use only)
\item[option$<$E$>$] The option data type that either be \lstinline+None+ or
\lstinline+Some(e)+, where \lstinline+e+ has to be of type E.
\item[process] Type of both actions and defined processes in histories.
\end{description}

\section*{method and parallel loop contracts}

Below, we list the keywords that can be used in contract.

\begin{description}
\item[given] Declare ghost in parameters.
\item[yields] Declare ghost out parameter.
\item[requires] State pre condition.
\item[ensures] State post condition.
\item[context] State a condition that is true before and after.
\item[invariant] State a condition that is true before, during, and after.
\item[modifies] List of locations that might be modified.
\item[accessible] List of locations that might be accessed.
\end{description}

If in/out paramters become supported then we will need ghost in/out parameters
as well. In all dialects of VAL, a single ghost parameter can be declared.
Many dialects support declaring multiple variables in one clause using
the same syntax as the programming language for variable declarations.

\section*{sequential loop contracts}

\begin{description}
\item[loop\_invariant]
\end{description}

\section*{proof statements}

\begin{description}
\item[assert] check is a property is true
\item[assume] assume the truth of the given property (TODO: argument must be boolean?)
\item[refute] prove that the property is not true
\item[inhale] add the given permissions and properties
\item[exhale] discard the specified permissions
\item[fold] Wrap permissions inside definition
\item[unfold] Unwrap a bundle of permissions
\item[create] Prove a magic wand formula
\item[qed] Conclude proof of magic wand
\item[use] extract permissions and/or reuse facts in a magic wand proof
\item[apply] Apply a magic wand
\item[witness] declaring witness names
\item[send . to .] Release permission to a future loop iteration
\item[recv . from .] Acquire permission from a previous loop iteration
\item[open] Open a definition in a predicate environment
\item[close] Close a definition in a predicate environment
\item[transfer] transfer permissions into and out of a CSL atomic block, consisting
of an atomic call, including the preceeding with and following then blocks.
\end{description}

\section*{expression decoration}

\begin{description}
\item[label] attach a label to an arbitrary expression
\item[with] to be executed just before the expression is evaluated
\item[then] to be executed just after an expression is evaluated
\end{description}


\section*{expressions}

\begin{description}
\item[**] Separating conjunction
\item[-*] Magic wand
\item[Perm] Field permission
\item[PointsTo] Field permission and value
\item[ArrayPerm] Quantifier free way of specifying access permissions for arrays.
\item[?x] Bind the name x to and output parameter
\item[???] Scaling operator for groups?
\item[Value] Declare an epsilon permission on a field
\item[unfolding . in .] Temporarily unfold definitions in (pure) expressions.
\item[$\backslash$old] refer to values in the 'old' state.
\item[$\backslash$result] refer to the result of a method call.
\end{description}

\section*{PVL specific constructs}

\begin{description}
\item[barrier] PVL only keyword to denote a kernel barrier
\item[fork] Fork off the run method of an object
\item[join] Join with the thread running teh run method of an object
\item[lock] lock what precisely?
\item[unlock] unlock what precisely?
\item[action] insert an action into a history
\item[create] create a history
\item[destroy] destroy a history
\end{description}



\end{document}

