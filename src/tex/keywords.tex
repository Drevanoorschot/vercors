\chapter{Keywords}

\section*{Special types}
\begin{description}
\item[frac] non-zero fractions
\item[zfrac] zero-able fractions
\item[resource] type of resource expressions
\item[classtype] type of type expressions
\item[seq$<$E$>$] Seqeunce or list of E
\item[set$<$E$>$] Set of E
\item[bag$<$E$>$] Bag of E
\item[cell$<$E$>$] Object that contains a single E with set and get (internal use only)
\item[option$<$E$>$] The option data type that either be \lstinline+None+ or
\lstinline+Some(e)+, where \lstinline+e+ has to be of type E.
\item[process] Type of both actions and defined processes in histories.
\end{description}

\section*{method and parallel loop contracts}

Below, we list the keywords that can be used in contract.

\begin{description}
\item[given] Declare ghost in parameters.
\item[yields] Declare ghost out parameter.
\item[requires] State pre condition.
\item[ensures] State post condition.
\item[context] State a condition that is true before and after.
\item[invariant] State a condition that is true before, during, and after.
\item[modifies] List of locations that might be modified.
\item[accessible] List of locations that might be accessed.
\end{description}

If in/out paramters become supported then we will need ghost in/out parameters
as well. In all dialects of VAL, a single ghost parameter can be declared.
Many dialects support declaring multiple variables in one clause using
the same syntax as the programming language for variable declarations.

\section*{sequential loop contracts}

\begin{description}
\item[loop\_invariant]
\end{description}

\section*{proof statements}

\begin{description}
\item[assert] check is a property is true
\item[assume] assume the truth of the given property (TODO: argument must be boolean?)
\item[refute] prove that the property is not true
\item[inhale] add the given permissions and properties
\item[exhale] discard the specified permissions
\item[fold] Wrap permissions inside definition
\item[unfold] Unwrap a bundle of permissions
\item[create] Prove a magic wand formula
\item[qed] Conclude proof of magic wand
\item[use] extract permissions and/or reuse facts in a magic wand proof
\item[apply] Apply a magic wand
\item[witness] declaring witness names
\item[send . to .] Release permission to a future loop iteration
\item[recv . from .] Acquire permission from a previous loop iteration
\item[open] Open a definition in a predicate environment
\item[close] Close a definition in a predicate environment
\item[transfer] transfer permissions into and out of a CSL atomic block, consisting
of an atomic call, including the preceeding with and following then blocks.
\end{description}

\section*{expression decoration}

\begin{description}
\item[label] attach a label to an arbitrary expression
\item[with] to be executed just before the expression is evaluated
\item[then] to be executed just after an expression is evaluated
\end{description}


\section*{expressions}

\begin{description}
\item[**] Separating conjunction
\item[-*] Magic wand
\item[Perm] Field permission
\item[PointsTo] Field permission and value
\item[ArrayPerm] Quantifier free way of specifying access permissions for arrays.
\item[?x] Bind the name x to and output parameter
\item[???] Scaling operator for groups?
\item[Value] Declare an epsilon permission on a field
\item[unfolding . in .] Temporarily unfold definitions in (pure) expressions.
\item[$\backslash$old] refer to values in the 'old' state.
\item[$\backslash$result] refer to the result of a method call.
\end{description}

\section*{PVL specific constructs}

\begin{description}
\item[barrier] PVL only keyword to denote a kernel barrier
\item[fork] Fork off the run method of an object
\item[join] Join with the thread running teh run method of an object
\item[lock] lock what precisely?
\item[unlock] unlock what precisely?
\item[action] insert an action into a history
\item[create] create a history
\item[destroy] destroy a history
\end{description}

