\chapter{Overview of the VerCors Language}

This chapter gives a quick overview of the main ingredients of the
VerCors language, including program and specification language, and a
few special proof hints. The next chapter illustrates the most
important features on several examples.
 
\section{The Syntax of PVL}

The syntax of PVL is similar to the syntax of Java.
The most important differences are the following.
\begin{itemize}
\item The visibility modifiers \lstinline+public, private, protected +, etc. are not allowed.
\item In principle, \lstinline+static+ members are not allowed.
\item Thread creation is Java like, but because PVL has no inheritance, it
is written \lstinline+fork t+. Likewise, thread joining
is written \lstinline+join t+ (see Section~\ref{sec:forkjoin} for an example).
\item There are some additional primitive types:
\begin{description}
\item[resource] the type of separation logic formulas
\item[frac] the type of permission fractions
\item[seq$<E>$] the type of lists (or sequences) over the type \(E\).
Similar to Java arrays, \(E\) can be a primitive type. (See
section \ref{list examples}\ for examples of list expressions.)
\item[set$<E>$] Sets of elements of type $E$.
\item[bag$<E>$] Bags of elements of type $E$.
\end{description}
\item There is a shorthand to define functions (\emph{i.e.}, query
  methods without side effects). They can be defined by giving their
  contract and
formal parameters, followed by the $=$ sign and their body. For
example, the \lstinline+fib+ function can be defined as follows:
\begin{verbatim}
requires n>=0;
static int fib(int n) = n<2 ? 1 : fib(n-1) + fib(n-2);
\end{verbatim}
Note that functions that do not depend on the field of an object are an
exception to the \lstinline+static+ rule:
without declaring such a function static, it cannot be proven that the same function
in two instances is actually the same function.
\item Abstract predicates are functions of type \lstinline+resource+.
Predicates are not allowed to have a contract. Because they are functions they can be static.
\end{itemize}

\section{Specification and Annotation Language}

\subsection{Formulas}

In addition to the standard operators from first-order logic, the
following constructs are available to define assertions.
\begin{description}
\item[Perm(loc,$\pi$)]: fraction $\pi \in (0,\ldots,1]$ access to  the location \lstinline+loc+. 
\item[PointsTo(loc,$\pi$,$v$)]: fraction $\pi \in (0,1]$ access to the location \lstinline+loc+, which contains the value $v$.
\item[Value(loc)]: infinitesimal read access to the location \lstinline+loc+.
\item[($\backslash$forall $\tau~v$ ; $\psi$ ; $\phi$ )]: boolean universal quantification: $\forall i : \phi \Rightarrow \psi$.
\item[($\backslash$forall* $\tau~v$ ; $\psi$ ; $\phi$ )]: separating universal quantification: $\bigstar i : \phi \Rightarrow \psi$.
\end{description}

The separation logic \(\star\) operator is written \lstinline+**+ in
PVL, to disambiguate it from multiplication.

Note that while the PVL language does not allow the \lstinline+final+ modifier,
you can get the same effect by using the \lstinline+Value+ predicate on a field.
That is, the fraction of a permission that has to be given up to create
a \lstinline+Value+ permission can never be recovered and hence 
write permission can never be recovered.

\subsection{Contract clauses}

Method contracts can use the following clauses:
\begin{description}
\item[requires $\phi$;] denotes a pre-condition $\phi$.
\item[ensures $\phi$;] denotes a post-condition $\phi$.
\item[given $\tau~v$;] declares a ghost formal in-parameter $v$ of type $\tau$.
\item[yields $\tau~v$;] declares a ghost formal out-parameter $v$ of type $\tau$.
\end{description}

For example, a method that takes some permission $p$ for a field \lstinline+x+
and returns possibly less permission $q$ can be specified as follows:
\begin{lstlisting}
  given    frac p;
  requires Perm(x,p);
  yields   frac q;
  ensures  Perm(x,q) ** q <= p;
  void drop() { ...
\end{lstlisting}

\subsection{Proof annotations}

In addition to the intermediate assertions, sometimes also proof hints
are necessary. The following proof hints are available.

\begin{description}
\item[assert $\phi$;]: assert the given resource property $\phi$.
\item[assume $\phi$;]: assume the given resource property $\phi$.
\item[unfold $f(\vec x)$;]: replace $f(\vec x)$ by its body.
\item[fold $f(\vec x)$;]: pack the body of $f(\vec x)$ into the
  function.
\item[\(<\)expr\(>\) with { ... }]: after the \lstinline+with+ clause,
  actual parameters can be given for the ghost in-parameters.  
\item[\(<\)expr\(>\) then { ... }]: after the \lstinline+then+ clause,
  ghost out-parameters can be assigned to locally visible variables.
\item[unfolding \(<\)predicate\(>\) in \(<\)expr\(>\)]: evaluate the expression,
while temporarily unfolding the predicate.
\end{description}


