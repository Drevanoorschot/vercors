\chapter{Installation and Use}

The VerCors tool is designed to leverage third-party back-end tools, such as Z3,
Boogie, Chalice and Silicon.
In order to be able to find multiple versions of these back ends, they have been packaged using
environment modules\footnote{\url{http://modules.sourceforge.net}}.
The VerCors tool has been packaged in two ways:
\begin{description}
\item[standalone] In this version the dependencies are hidden from the user
and you can run the tool by typing the full path to the tool script.
For Linux/MacOSX/etc. this is \verb+unix/bin/vct+ and for Windows
it is \verb+windows/cmd/vct+.
\item[modular] In this version, the dependencies and the tool are all part of a
directory with modules. When you use the provides modules, you get command
line access to not only the tool itself, but also to the back ends.
\end{description}

Note that if you still have your old System Validation Virtual Machine
image, you can choose to run the VerCors tool inside it.

\section{Overview of Dependencies}

The VerCors tool and its back ends depend on the run times for several programming languages.
The installation of these is relatively straightforward and instruction are easy to find
on the web\footnote{Installing mono from source on Linux can be tedious.}.
\begin{description}
\item[java 7] \textbf{(required)} Note that the tool will not work with earlier version of Java than 7.
It should work with Java 8, but this is not actively tested.
For new versions of Java visit \url{http://www.oracle.com/technetwork/java/index.html}.
\item[tcl] \textbf{(required)}
\begin{itemize}[topsep=-1em]
\item Windows users should get this from \url{http://www.activestate.com/}.
\item For MacOS this is installed as soon as you install the Xcode development environment.
\item On Linux systems this is most likely already installed and otherwise it is available as a standard package.
\end{itemize}
\item[C\# runtime] \textbf{(optional)} For users who want to run Boogie as a back-end or run a back-end
that depends on Boogie, the C\# runtime is required.
\begin{itemize}[topsep=-1em]
\item On the Windows platform this is available natively.
\item On MacOS, you need to install
the mono runtime or sdk. Note that you need at least version 2.10. You can download and install the latest version from the developers website: \url{http://www.mono-project.com/}.
\item On Linux, you need to install
the mono runtime or sdk. You may be able to get it using
your package manager. If that version is too old, you may have to
compile from sources.
\end{itemize}
\item[C preprocessor] \textbf{(optional)} For users that wish to verify examples written
in C, OpenCL, etc., the installation of a C pre-processor is required.
%\footnote{Note, this is not necessary for the Program Verification assignments}
The default pre-processor that is used is the one that is part of the
LLVM C compiler: clang.
\begin{itemize}
\item Windows users can get it from \url{http://clang.llvm.org/}. 
\item MacOS users get it when they install Xcode.
\item On Linux it is most likely available separately or as part of an
  llvm installation.
\end{itemize}
\end{description}

Note, that for the Program Verification assignments none of the optional
dependencies are needed.

\section{Downloads}

The back-ends on which the tool relies need careful scripting to make them work.
Therefore, we have packaged them in the zip archives. There are two forms
in which they can be downloaded:
\begin{description}
\item[vercors-standalone.zip] Stand alone variant of the VerCors tool and back-ends.
\item[vercors-modular.zip] Modular variant of the VerCors tool and back-ends.
\end{description}

\subsection{Windows}

\begin{enumerate}
\item Create a folder \verb+fm-tools+ in your home directory and unzip \verb+vercors-standalone.zip+ in
that folder.
\item From the command line issue the command:
\begin{verbatim}
path C:\Users\user\fm-tools\vercors\pvl\windows\bin;%PATH%
\end{verbatim}
You can add this to your default settings by changing the PATH environment variable.
\end{enumerate}


\subsection{Linux and MacOSX, using PATH}

The easiest way of using the tools is to add them to you PATH:

\begin{enumerate}
\item Create a folder \verb+fm-tools+ in your home directory and unzip unzip \verb+vercors-standalone.zip+ in
that directory.
\item add the line
\begin{verbatim}
export PATH=$HOME/fm-tool/vercors/pvl/unix/bin:$PATH
\end{verbatim}
to the file \verb+~/.bashrc+.
\item Start a new shell.
\end{enumerate}

\subsection{Linux and MacOSX, using modules}

If you already already use environment modules then you can use
the module files for all of the tools:

\begin{enumerate}
\item Create a folder \verb+fm-tools+ in your home directory and unzip \verb+vercors-modular.zip+
in that directory.
\item add the line
\begin{verbatim}
module use $HOME/fm-tools/modules
module load vercors/pvl
\end{verbatim}
to the file \verb+~/.bashrc+.
\item Start a new shell.
\end{enumerate}


\subsection{Testing the installation}

\begin{enumerate}
\item Go to the directory \verb+fm-tools/vercors/pvl/examples+.
\item Check that the following command produces a long list of options:
\begin{verbatim}
vct --help
\end{verbatim}
\item Next try testing Silicon by issuing the command:
\begin{verbatim}
vct --test=silicon
\end{verbatim}
This should produce quite some output, ending in:
\begin{verbatim}
==============================================
While running class vct.main.SiliconInstallationTest: 0 failures out of xx tests
==============================================
all test suites have been succesful.
\end{verbatim}
\end{enumerate}


\section{Command line options}

The tool has many options; here we highlight the most important ones.

\begin{description}
\item[-\,-silver=silicon] Selects the main stream Silicon back end that can verify separation logic without \lstinline+\forall*+
\item[-\,-silver=silicon\underline{~}qp] Selects the experimental Silicon back end that can verify separation logic
and un-nested \lstinline+\forall*+ quantifiers.
\item[-\,-progress] Will show the various passes of the verification process.
\item[-\,-silicon-z3-timeout=$t$] Set the timeout for Z3 in Silicon. The default for this is half a minute (30000ms).
Decrease the value to find errors/failures quickly, but get false negatives on some true statements.
Increase to reduce the amount of false negatives.
\item[-\,-inline] This option inlines non-recursive predicate definitions
before doing the actual verification. By using this option, you get
the benefit of concise notation without the need to insert many fold and unfold
annotations.
% \item[-\,-explicit] Instructs the verifier to encode predicates with arguments
% using witness objects.
% \item[-\,-witness-inline] When verifying permissions only this option may be omitted.
% When using witnesses for full functional verification this option is needed
% to work around a bug that causes the tool to forget necessary information about getters.
\end{description}

\section{Practical Use}

\subsection{Error Messages}

The error messages returned by the tool can be cryptic, therefore we
give an overview of how to decrypt them.

\paragraph{Failures}

When the tool has a failure due to a user error, then the message should point out where
the error has been detected and what the error is. For example, when trying to
assert that \verb+1+ is equal to \verb+true+, the tool says:
\begin{verbatim}
At file err.pvl from line 4 column 11 until line 4 column 17:
 Types of left and right-hand side argument are uncomparable:
  Integer/Boolean
\end{verbatim}

\paragraph{Aborts and Crashes}

When the tool aborts due to an internal error or when it crashes
completely, there will be
a stack trace. If you are lucky, the line(s) just before the start of the stack trace
will tell you why the tool crashed.

\subsection{Bug Reports}

Bug reports can be sent by email to Stefan Blom (\verb+s.c.c.blom@utwente.nl+) with the following information:
\begin{itemize}
\item The platform used (Windows, MacOSX, Linux).
\item The command line used, which must include the \verb+--progress+ option.
\item The input files.
\item The output of the tool.
\end{itemize}
Note that if you take the effort to reduce your problem to a bug report containing a small example that shows the problematic behavior,
it will greatly improve your chances of getting an answer and/or bug fix fast.

\subsection{Tips and tricks.}

\begin{itemize}
\item Do not try to prove everything at once. Small errors can lead to timeouts,
which do not tell you where the error is. Whenever possible, add your
requires and ensures clauses one-by-one to see which one causes the problem.
\item When reasoning about a program in your mind, you will believe that
certain properties are true. It is a good idea to write those properties as assertions.
This will help you find where and why the tool fails. In some cases, it can solve
problems where the prover cannot prove a result directly, but it can prove the assertion
and from the assertion it can prove the result.
\item If a program has multiple branches (if-then-else and/or while) then it can be useful
to temporarily disable checking one or more branches. The annotation
\begin{verbatim}
assume false;
\end{verbatim}
is the logical equivalent of stating {\em do not check anything that follows.}
\item Sometimes the tool reports Pass when you are certain that there is a mistake.
To check if the place in the code where the mistake is, is reachable, you can insert
\begin{verbatim}
assert false;
\end{verbatim}
If this line is reachable then either the tool will report that the assertion might not be true
or there will be a time out. If the line is not reachable (or there is a bug in the tool)
then the result will still be Pass.
\end{itemize}

