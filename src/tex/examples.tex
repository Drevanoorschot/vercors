
\chapter{Examples}

This chapter introduces and explains the specification features of the VerCors
tool set step-by-step, by means of a series of example
% The fractions used in this chapter are in the style of Chalice, which means that they are integers
% in the range $[1,..,100]$, which can be thought of as percentages. Thus, the full 100\% grants write permission
% and any thing less is read-only.

\section{Fractional Permissions}

\begin{listing}[b]
\begin{tabular}{l@{~~~~~}l}
\lstinputlisting{permissions.pvl}
&
\lstinputlisting{main.pvl}
\end{tabular}
\caption{Fractional permissions.}
\label{fractional permission}
\end{listing}

\paragraph{Permissions}
The file \verb+permissions.pvl+ (See \listref{fractional permission}) contains an example of a counter class
with an increment method and a main method. The increment method has a simple contract that requires
write permission for the field \lstinline+x+, and ensures both write permission
and the fact that the value afterwards is the same as the old value plus 1.

In the main method, we instantiate a counter, then set it to 16, increment it by 1 and assert that it is 17.
This assertion is correct. Next, we increment the counter again and repeat the assertion, which this time
is incorrect.

To verify these two classes using the VerCors tool set, we call the
tool from the command line as follows:
\begin{verbatim}
vct --silver=silicon permissions.pvl main.pvl
\end{verbatim}
The result will be something like
\begin{verbatim}
Errors! (1)
=== main.pvl                       ===
   6     assert c.x==17;
   7     c.incr();
        [---------------
   8     assert c.x==17;
         ---------------]
   9   }
  10 }
-----------------------------------------
  assert.failed:assertion.false
=========================================
\end{verbatim}
indicating that the second assertion is wrong.

\begin{listing}
\lstinputlisting{loop.pvl} 
\caption{Loop invariant.}
\label{loop invariant}
\end{listing}
\paragraph{Loops}
The file \verb+loop.pvl+ (see \listref{loop invariant}) implements 
an increase method for the counter class, which increases a counter multiple times using a loop.
The loop invariant has three components: 
\begin{itemize}
\item a boolean component specifying the invariant
for the loop counter \lstinline+i+, 
\item a resource component specifying that \lstinline+c+ is not null and that \lstinline+c.x+ can be written, and
\item another boolean component specifying the functional invariant on
  the counter.
\end{itemize}
Note how the first invariant uses boolean conjunction to connect two boolean expressions.

The loop is verified by issuing the following command:
\begin{verbatim}
vct --silver=silicon permissions.pvl loop.pvl 
\end{verbatim}

\section{Ghost Parameters}

\begin{listing}
\begin{tabular}{l@{~~~~~}l}
\lstinputlisting{parameters1.pvl}
&
\lstinputlisting{parameters2.pvl}
\end{tabular}
\caption{Ghost Parameters}
\label{ghost parameters}
\end{listing}


So far, we have used full write permissions only.
When considering arbitrary permissions, the requirements
and the guarantees become parametric in 
the fraction. That is, specifications can have both permission
arguments (ghost parameters), declared with the \lstinline+given+
keyword and (ghost) permission return values, declared with the \lstinline+yields+
keyword. For example, in \listref{ghost parameters}, we show
two implementations of a drop method, whose contract states that
it looses permissions.

To pass a ghost parameter during a method invocation,
one can append a block with assignments to an invocation
using the \lstinline+with+ keyword (as in the call to \lstinline+drop+
in the method \lstinline+main+. Each left-hand side
of an assignment must be a ghost parameter of the method
and the right-hand side is the value passed.
The function drops permissions, so the tool will complain
that the second write might not be possible when
the example is verified using the command

\begin{verbatim}
vct --silver=silicon permissions.pvl parameters1.pvl
\end{verbatim}

To obtain ghost return values, another block with assignments
can be attached using the \lstinline+then+ keyword.
In this case the left-hand sides are locations that must be writable
from the calling site in the program, while the right-hand sides
may use the ghost return values as part of expressions
that are assigned. This is demonstrated in the twice method
on the right-hand side, which can be validated with:

\begin{verbatim}
vct --silver=silicon permissions.pvl parameters2.pvl
\end{verbatim}

\section{Functions and Predicates}

\begin{listing}
\lstinputlisting{functions.pvl}
\caption{Functions and Predicates}
\label{functions and predicates}
\end{listing}

Methods can have side effects and therefore method invocations
cannot be used in specifications in general. A special class
of method, called functions, can be used inside specifications.
Essentially a function is a method that does not have side effects. A special role is played by functions
that return an expression of type resource, the so-called \emph{predicates}.

In \listref{functions and predicates}, we show an example of
a counter in which the permission to access the field \lstinline+x+
is wrapped inside a predicate \lstinline+state+.
The value of \lstinline+x+ is accessed through a getter 
function \lstinline+get+. Note how the required predicate \lstinline+state()+
is unfolded to be able to access \lstinline+x+. There is no need to
put \lstinline+state()+ in the post condition, because functions by design do not
change the state, including the way in which permissions are folded.
It is possible for functions to have boolean post-conditions.

Also note how at the end of the contructor,
the \lstinline+fold+ keyword is used to wrap the permission
to write \lstinline+x+ into the \lstinline+state+ predicate and how the increment
method has to unwrap this with the \lstinline+unfold+ instruction.

\section{The data type list}\label{sec:list}
\label{list examples}

\begin{listing}
\lstinputlisting{list.pvl}
\caption{Using the list data type}
\label{list}
\end{listing}

In \listref{list}, we show how to use the native sequence or \lstinline+list+ datatype
to reason about the contents of a linked list. Note that because the tool
cannot do type inference, we need to write the element type explicitly
when writing list constants.


\section{Working with arrays}\label{sec:arrays}

In \listref{array copy}, we show the implementation of array copy complete with the full functional specification.
Note that we do not use a function to compute the contents of an array. This is because
there is a bug in Silicon that renders it unable to use the ensures clause of a function
that requires a quantified permission.

\begin{listing}
\lstinputlisting{array-all.pvl}
\caption{Full functional correctness of array copy}
\label{array copy}
\end{listing}

\section{Parallel programs with fork/join}\label{sec:forkjoin}


\Listref{fibonacci} shows an example of a parallel computation
in PVL. We compute the Fibonacci numbers by creating a Fibonacci object
and then calling the run method. If the input is less than 2 then the result is
1 and immediately written to the output. Otherwise two Fibonacci objects are created: one for \lstinline+input-1+ and
one for \lstinline+input-2+. Those object are then forked, which causes the system
to spawn threads that execute the run methods. Next, these two threads are joined, in order to wait for the result.
The \lstinline+fork+ statement in PVL essentially works like the \lstinline+start+ method in the
class \lstinline+Thread+ in Java. The \lstinline+join+ statement works like the corresponding
method in the class \lstinline+Thread+ in Java. Every object in PVL that implements a \lstinline+run+ method
should be thought of as extending the \lstinline+Thread+ class in Java.

\begin{listing}
\lstinputlisting{fibonacci-all.pvl}
\caption{Parallel Fibonacci}
\label{fibonacci}
\end{listing}

\section{Using locks in PVL}

Simple non-reentrant locks can be used in PVL.
To be able to lock objects belonging to a certain class:
\begin{enumerate}
\item The class must define the predicate \lstinline+lock_invariant()+.
\item The class must have a constructors, and the lock
invariant must hold at the end of the constructor. (If it does not, the tool
will complain that it cannot fold because there is an implicit fold at the end of the constructor.)
\item The syntax for locking and unlocking an object \lstinline+o+ is \lstinline+lock o;+
and \lstinline+unlock o;+, respectively.
\end{enumerate}


As an example, we show the classical Owicki-Gries example: two threads are spawned and each
thread increments a global variable by one. The main program code can be found in
\listref{og main} and the code for the worker in \listref{og worker}.

\begin{listing}
\lstinputlisting[linerange={2-481}]{OwickiGries.pvl} 
\caption{Owicki-Gries main code.}
\label{og main}
\end{listing}

\begin{listing}
\lstinputlisting[linerange={2-481}]{Worker.pvl} 
\caption{Owicki-Gries worker code.}
\label{og worker}
\end{listing}

The PVL language also has \lstinline+wait o;+ and
a \lstinline+notify o;+ statements. These statements behave as the Java \lstinline+wait+
and \lstinline+notifyAll()+ statements. Moreover, they require that the lock is held:
\lstinline+held(o)+. (See files in \verb+examples/waitnotify+.)

\section{Compiling and Running PVL Programs}

An experimental compiler for PVL programs exists. The compiler
translates PVL to Java, which allows PVL code to be called from Java.
For example, the code in \listref{og test} instantiates and calls
the Owicki-Gries main method.
To compile the program, run the (unix only) \verb+pvlc+ script:
\begin{verbatim}
pvlc OwickiGries.pvl TestOwickiGries.java Worker.pvl
\end{verbatim}
This will translate the PVL files into Java and then it will compile
both the generated and hand-written Java files.
To run a program use the \verb+pvl+ script:
\begin{verbatim}
pvl TestOwickiGries
\end{verbatim}

\begin{listing}
\lstinputlisting[linerange={2-481}]{TestOwickiGries.java} 
\caption{Owicki-Gries test code.}
\label{og test}
\end{listing}


